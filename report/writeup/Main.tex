\documentclass{article} %specifies the document being created is an article
% \documentclass[12pt]{article} %specifies the document being created is an article
% \documentclass{COMPXXXX}
\usepackage{graphicx} %tells LaTeX we want to include images in the document 
\usepackage{natbib} %this allows you work on the references and citations in Harvard format
\usepackage{import} %this allows you to import other files into the document
\usepackage{array}
\usepackage{blindtext}
\usepackage{titlesec}

\usepackage{geometry}
\geometry{
    margin=1in, % 2.5cm CIS = 1in
    includeheadfoot
}
% \usepackage{setspace}
% \onehalfspacing
% \usepackage{fontspec}
% \setmainfont{Times New Roman}

% \setcounter{secnumdepth}{2}
% \setcounter{tocdepth}{2}

\begin{document} %start the document

\import{./Chapters/Introductory/}{TitlePage.tex}
\import{./Chapters/Introductory/}{AbstractAndPreface.tex}
\import{./Chapters/Introductory/}{Acknowledgements.tex}
\import{./Chapters/Introductory/}{Index.tex}

\newpage
% Each chapter should start on a new page. Chapter headings should appear more important than section headings. The following usually have one or more chapters devoted to them.
% The introduction should set the scene for the project, and give the background to the project. It should also state the objectives of the project.
\import{./Chapters/MainBody/Introduction/}{Background.tex}
\import{./Chapters/MainBody/Introduction/}{AimsAndObjectives.tex}
\import{./Chapters/MainBody/Introduction/}{Approach.tex}
\import{./Chapters/MainBody/Introduction/}{FrameworkJustification.tex}

\newpage
% The literature review should describe the current state of the art in the area of the project, and what work has been done in the area.
\import{./Chapters/MainBody/LiteratureReview/InitialResearch/}{Approach.tex}
\import{./Chapters/MainBody/LiteratureReview/InitialResearch/}{Problem.tex}
\import{./Chapters/MainBody/LiteratureReview/InitialResearch/}{LiteratureReview.tex}
\import{./Chapters/MainBody/LiteratureReview/InitialResearch/}{Conclusions.tex}

\newpage
\import{./Chapters/MainBody/LiteratureReview/ProductResearch/}{ExistingSolutions.tex}
\import{./Chapters/MainBody/LiteratureReview/ProductResearch/}{Considerations.tex}
\import{./Chapters/MainBody/LiteratureReview/ProductResearch/}{Requirements.tex}

\newpage
\import{./Chapters/MainBody/LiteratureReview/DesignProposal/}{Approach.tex}
\import{./Chapters/MainBody/LiteratureReview/DesignProposal/}{Prototypes.tex}
\import{./Chapters/MainBody/LiteratureReview/DesignProposal/}{Reflection.tex}

\newpage
\import{./Chapters/MainBody/}{AnalysisAndRequirements.tex}
\import{./Chapters/MainBody/}{DesignPhilosophy.tex}

\newpage
\import{./Chapters/MainBody/IntegrationAndTesting/}{Integration.tex}
\import{./Chapters/MainBody/IntegrationAndTesting/}{Testing.tex}
\import{./Chapters/MainBody/IntegrationAndTesting/}{ProductEvaluation.tex}

\newpage
% The closing chapters commonly include a summary and a conclusion together with any recommendations. In summarizing, highlight the important stages and outcomes of the project. The conclusions would normally consider and comment critically upon the results of the project; this includes both the process and the product.  This should include a consideration of the extent to which the aims of the project have been achieved. Finally, recommend ways in which the work could be applied or extended.
\import{./Chapters/ClosingChapters/}{SummaryAndConclusion.tex}
\import{./Chapters/ClosingChapters/}{References.tex}
\import{./Chapters/ClosingChapters/}{Appendices.tex}

\end{document} %end the document
