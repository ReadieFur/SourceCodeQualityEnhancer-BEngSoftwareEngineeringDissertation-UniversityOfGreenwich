% Summary
% What were its flaws?
% Did it do what I wanted it to do?
% Conclusion
% How did the project go?
% What did I learn?
% What would I do differently next time?
% How could the project be taken further?
\subsection{Summary}
From the research conducted and the results that were obtained, it can be seen that the project has been successful in achieving the goals that were set out. The project has been able to create a tool that can restructure source code in a way that is easy to use and can be integrated into a user's workflow. The tool has been designed to be fast and efficient, and can be run in a variety of ways to suit the user's needs. The tool has been tested and evaluated, and has been found to perform well in a variety of scenarios. The project has also been able to demonstrate the benefits of using a tool like this, and has shown that it can save time and improve the quality of code. Overall, the project has been a success and has achieved the goals that were set out at the beginning.

\subsection{Conclusion}
The project has been a valuable learning experience, and has provided a good opportunity to develop new skills and knowledge. The project has allowed me to gain a better understanding of the principles and patterns of object oriented programming, and has given me the opportunity to put these into practice. The project has also allowed me to develop my skills in software development, and has given me the opportunity to work on a real-world project. The project has been challenging at times, but has been rewarding and has provided me with a sense of accomplishment. Overall, the project has been a positive experience, and has helped me to develop my skills and knowledge in a variety of areas.
In the future I would like to continue working on this project, and would like to further develop the tool and add new features and functionality. A primary goal that I would like to achieve for this tool is to make it multi-language compatible, as this would greatly increase the usefulness of the tool and make it more accessible to a wider range of users. Ideas of how such a change could be implemented is to provide a language schema in the configuration that would then be used to parse to source code into a universal AST that could then be used to restructure the code. This would also allow for a much more flexible tool that could be used on multiple languages and platforms, as currently the tool is limited to only working with C\# code via the MSBuild pipeline which is limited to the Windows platform, the integration of a custom intermediatory AST parser would allow for the tool to be used on any platform and with any language.
