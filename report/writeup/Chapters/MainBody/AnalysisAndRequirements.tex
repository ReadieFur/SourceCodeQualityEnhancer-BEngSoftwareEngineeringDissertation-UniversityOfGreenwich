% Analysis and requirements specification
% This chapter should describe the analysis and requirements specification of the project.
% What is needed to complete the project?
% What are the requirements of the project?
% What are the functional requirements of the project?
% What are the non-functional requirements of the project?
% What are the design requirements of the project?
% What are the testing requirements of the project?
% What are the evaluation requirements of the project?
\subsection{Evaluation}

In order to evaluate this tool the following will be assessed:
\begin{itemize}
	\item Function.
	\item Comprehensibility.
	\item Performance.
\end{itemize}
The function of the reproduced code should maintain the result of the original code. This can be measured by running the original code and the reproduced code either through manual or automated testing and then comparing the results.

The comprehensibility of the reproduced code should be no harder to digest than the original code. This is a harder metric to measure as it is subjective, so in order to evaluate this peer review can be used as well as tools that measure cyclomatic complexity.

Finally, the performance of the reproduced code should be on-par or better than the original code. This can be measured by running the original code and the reproduced code and comparing the time taken to run each. While this tool will not aim to optimize the performance of the code, it should not make it worse.

In the paper by \cite{8681007}, it was said that "no tool succeeds in all respects". By this they mean that no tool is perfect and that each tool has its own strengths and weaknesses. This statement is something that should be kept in mind with this project as it will not aim to create a tool that succeeds in all respects. Instead it will aim to create a tool that can combine the strengths of each tool to create a more complete tool.
