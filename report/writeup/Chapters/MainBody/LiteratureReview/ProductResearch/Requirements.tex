% Requirements
% Analysis of requirements
% Discuss the requirements of the project, and what is needed to complete the project.
% Comparison of systems
% For some projects there may be similar existing systems. This would be a good place to identify these, compare them and elicit requirements from them.
% Requirements elicitation
% Perhaps you have conducted some actual research with questionnaires and interviews and such like (having obtained research ethics clearance of course). This would be a good place to discuss this.
% Functional requirements
% A clear statement of functional requirements is the starting point for your design. Without this you will look foolish.
% Non-functional requirements
% Clear statement of non-functional requirements. It may be an idea here to refer back to your earlier LSEPi section as compliance is likely to (and should) be an important aspect here.
\subsubsection{Requirements}
% Establish what quality software means.
% How should I monitor and measure my goals.
% What can I follow as a guide to ensure I am on the right track (e.g. existing tools, gantt charts, etc).
% Use of a widely accepted programming language.
% For the gantt chart instead of displaying tasks on the timeline, put a preliminary task and use the CI pipeline to update the gantt chart with the tasks that are being worked on.

For the development of this project, a multitude of base requirements should be laid out so that project has something to follow. These requirements will be used to guide the development of the project and ensure that the project is on track to meet the goals that have been set out.
Because the CI pipeline will be used for this project, a continuously updating todo list will be used, this list will change with new features and get marked off as started, completed or dropped as the project progresses. By having an adaptive todo list it allows for the project to be more flexible and allows for the project to be more easily adapted to changes in requirements, goals or new discoveries that may be made during the development of the project. The downside to such an approach is it is less structured and so it can be harder to follow and find a set goal, however, it is still suitable for this project as the goals and features are modular, as discussed in the design philosophy section, how there is no set way to restructure code and so the tool will be updated with new features as they are discovered or needed.
Another major requirement is the analysis of how to measure the current progress of the project through time management. While this is not as easy to do due to the nature of a CI pipeline, it is still important to have a rough idea of how long a task will take to complete and how much of the alloted time should be spent on certain tasks. For this a preliminary Gantt chart will be used to aid in the time management of the project, this will be updated as the project progresses, it is important to follow this Gantt chart, otherwise if the project time went unmanaged, it may result in incomplete or worse, rushed features that are not up to the standard that is expected due to spending too much time on certain tasks and not leaving enough time for others.
