% Legal, social, ethical, professional issues and considerations
% You may find it helpful to include a list of all applicable laws and perhaps other laws that you have considered and excluded, with reasons why they do not apply, if appropriate. It is generally not a good idea to decompose this section into sections labelled as ‘Legal’, ‘Social’, and so on as these aspects are intimately related.
\subsubsection{Considerations}
% What do I think software quality should be defined by?
% How do I think software quality could be measured?
% Would research into other areas help?
% Discuss the thought of making the tool multi-language compatible.
% Discuss limitations of:
% - The tool
% - The research
% - Time
% - Knowledge

Before undertaking the development of this project, a number of considerations must made to ensure that the project would be successful. A first consideration was the choice of programming language. It has been decided that the project would be developed in and for the C\# language as this is a widely used language and so it is a valid candidate for this project. While it would be desirable to create a tool that supports multiple languages, this would have a significant impact on the complexity of the project and so it was decided that the project would be limited to C\# due to the time restraints of the project in addition to my knowledge of other languages not being as strong as my knowledge of C\#.
As previously discussed, software quality does not have a set-in-stone definition, so in order to measure the quality of the software that is produced a baseline of things will be measured, such as consistency of the code, maintainability through the use of up-to-date language features, the use of object oriented programming principles and design patterns, and the ease of readability of the code. These are all important factors that can be used to measure the quality of the software that is produced in this project.
