% Product research
% Comparing of similar products or platforms against a set criteria for Usability purpose.
\subsubsection{Related work}

\cite{8681007} covers how there are many tools that have been created to measure software quality. Some existing tools used to aid developers in maintaining particular coding styles include, but are not limited to:
\begin{itemize}
	\item Lint4j - Checks the performance of code;
	\item Checkstyle \& Codacy - Indicates errors and flags when language conventions are not followed;
	\item PMD - Inspects for code duplication;
\end{itemize}
These tools can be very helpful when used in conjunction with each other, however with the sheer number of tools available it can become difficult to decide which tools to use \citep{6606742}.

\subsubsection{Defining software quality}
\cite{6606742} said that "Software Engineering (SE) has very peculiar characteristics that strongly relate it to social sciences that encourage the implementation of empirical studies that are able to assess the effectiveness of techniques, methodologies and processes proposed in the area". This can interpreted this as meaning that software quality is a broad topic and that the definition of what is considered high quality software can vary from person to person. In the paper by \cite{10.1145/3428029.3428047}, it is inferred that teachers define high quality code by how readable the source code is to other developers. \cite{10.1145/3428029.3428047, 10.1145/2674683.2674702} defined code quality as what can be determined by "just looking at the source code, i.e. without checking against the specification".
% TODO: What am I using to define software quality? And why?
