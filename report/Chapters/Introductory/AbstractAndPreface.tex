\subsection*{Abstract}
\addcontentsline{toc}{subsection}{\protect\numberline{}Abstract}
% Abstract
% Occupying less than half a page, is a short description of the intention of the project.
This paper considers the current state of source code quality and the importance of maintaining a high quality codebase. It discusses how there is a lack of uniformity in coding styles and how this can lead to a decrease in efficiency within development teams and thereby cause increased costs and time to develop a project. The paper also shows an in-depth implementation of a program that can aid in increasing software quality and how it can be used to reformat source code to a user defined style with support for object oriented programming principles in combination with design patterns to create a well structured and higher quality codebase from a pre-existing codebase, or from scratch. Throughout this paper the use of the spiral software development methodology is used as it suits the project area well.

\subsection*{Preface}
\addcontentsline{toc}{subsection}{\protect\numberline{}Preface}
% Preface
% Describe the context of the project, and the main idea behind work, without giving details.
This project was undertaken as part of the requirements for the degree in BEng Hons, Software Engineering at the University of Greenwich. The motivation to researching this area comes from an interest in developing tools that aid programmers in creating software as individuals and in a team, so that uniformity can be maintained, and increasing the readability of a program that may one day be discontinued or passed onto another developer or set of developers.
