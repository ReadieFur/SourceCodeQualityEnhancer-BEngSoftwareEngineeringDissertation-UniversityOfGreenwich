% Summary
% What were its flaws?
% Did it do what I wanted it to do?
% Conclusion
% How did the project go?
% What did I learn?
% What would I do differently next time?
% How could the project be taken further?
\subsection{Summary}
From the research conducted and the results that were obtained, it can be seen that the project has been successful in achieving the goals that were set out. The project has been able to create a tool that can restructure source code in a way that is easy to use and can be integrated into a user's workflow. The tool has been designed to be fast and efficient, and can be run in a variety of ways to suit the user's needs. The tool has been tested and evaluated, and has been found to perform well in a variety of scenarios. The project has also been able to demonstrate the benefits of using a tool like this, and has shown that it can save time and improve the quality of code. With more time and resources further benefits and efficiencies may have been achievable but given the constraints, the project has been a success and has achieved the goals that were set at the outset. It also provides a basis for further development and enhancements if desired.

\subsection{Conclusion}
The project has been a valuable learning experience, and has provided a good opportunity to develop new skills and knowledge. A better understanding of the principles and patterns of object oriented programming as been gained and provided the opportunity to put these into practice. The project has also allowed the development of skills in software development, and has provided me the opportunity to work on a real-world project. While challenging at times, it has been rewarding and has provided a sense of accomplishment, and has been a positive experience.

In the future with more time the aim would be to continue working on the project, to further develop the tool and add new features and functionality. The primary goal that would be desireable wold be for this tool is to make it multi-language compatible, as this would greatly increase the usefulness of the tool and make it more accessible to a wider range of users. An idea of how such a change could be implemented would be to provide a language schema in the configuration that would then be used to parse to source code into a universal AST that could then be used to restructure the code. This would also allow for a more flexible tool that could be used on multiple languages and platforms, as currently the tool is limited to only working with C\# code via the MSBuild pipeline which is limited to the Windows platform. The integration of a custom intermediatory AST parser would allow for the tool to be used on any platform and with any language. However, these are all developments that would enhance the tool created and this is only possible as the project has demonstrated that the use of a properly developed tool does allow for creating software quality and reformatting source code, which has been achieved here on the constrained project. It is exciting to see this succeed and know how this could further be improved.
