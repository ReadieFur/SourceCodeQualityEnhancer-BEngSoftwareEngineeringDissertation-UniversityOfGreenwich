\subsubsection{Document formatting}
The tool also provides various document formatting options to pick from, these options are designed to help the user maintain a consistent formatting style throughout their document. The tool provides the following formatting options: brace location, indentation, comment style, punctuation and object structure. All of these options make use of the abstract syntax tree (AST) that is generated for each document. The AST contains all information about the document in a programmatic way as provided by the Roslyn API. We have to navigate and interpret the AST as necessary for each of the formatting options.

\subsubsubsection{Comments}
Comments can be styled in various ways, the tool provides options to force comments to be on their own lines, or inline with code, and additionally choose wether comments should have a leading space or a trailing full stop. Each item in the AST contains a relative location to the document, using this location span we can determine what line the comment is on, if the user has configured comments to be on a new line then we can navigate backwards through the AST to check if there are any non-whitespace tokens on the same line as the comment, if there are then we can insert a new line token before the comment and add any necessary indentation. We can make some minor performance improvements by not checking tokens after the comment as single-line comments defined by using the trivia \texttt{$\backslash$$\backslash$} will always be the last non-whitespace token a line.
% TODO: Show how the AST is navigated in reverse to determine the line structure.

\subsubsubsection{Indentation}
To determine indentation levels for a given line the document is navigated line-by-line. A last-in-first-out (LIFO) stack is created to hold information about the current context "level". A context level is determined by the following tokens: \texttt{\{ \} ( ) [ ]}. When a token is found that increases the context level, the current context level is pushed onto the stack, and when a token is found that decreases the context level, the stack is popped. The indentation level is then determined by the number of items in the stack multiplied by the user-defined indentation size. This method allows for the tool to determine the correct indentation level for each line in the document.
We cannot do a simple text search for the tokens as they may be part of a string or comment, therefore we have to navigate the AST to find the tokens. Additionally the indentation level cannot be properly be calculated if the stack contains an invalid sequence of tokens, for example, if the stack contains \texttt{\{ [ \}}, the indentation level cannot be calculated as the stack is in an invalid state. However this should not occur as our custom analyzers are configured to only be run on syntactically valid C\# code.
% TODO: Show how the AST is navigated to determine the indentation level.

\subsubsubsection{Punctuation and brace location}
The punctuation sub-module allows the user to configure properties of language punctuation. In C\# punctuation is any token that is not a keyword or object variable, for example \texttt{; + -} etc. The punctuation sub-module allows the user to configure the spacing around punctuation, and wether punctuation should be on the same line as the previous token or on a new line. Within the YAML configuration file the user has the ability to define as many punctuation rules as they desire for the following options: space around and new line. The tool will then navigate the AST to find any matching punctuation and adhere it to the configuration. For each of these options within the YAML file, the user should provide an array of configurations, each of these configurations are to contain options for the formatter, a severity and a list of tokens that the rule should be applied to (see figure \ref{fig:YAMLPunctuation}).
% TODO: Show the YAML configuration for punctuation.

The brace location works very similar way to the punctuation, however the brace location has to work slightly differently for the closing brace because the closing brace, if configured to be on a new line, must have 1 less indentation than what the tokens would otherwise have.

\subsubsubsection{Object structure}
The object structure sub-module is rather minimal but can have a major impact on code readability, it provides the user with the option to force object properties and fields to be at the top of the object declaration with all methods at the bottom. This has a huge impact on readability and debugging as it contributes massively to reducing the amount of spaghetti code. Spaghetti code is code that is difficult to read and understand due to poor structure, this can be caused by having methods and properties mixed together in a class, by separating out the methods and properties it makes it far easier to understand what members a class has as there is no need to randomly dig through a class or struct to find potentially tucked away properties and fields between large method bodies.
Like before the sub-module will traverse the abstract syntax tree, if any object declarations are encountered, a sub-loop will be run to check the order of the object members, if for example a method declaration is found and further down the object declaration a property is found, the tool will create a diagnostic and return it to the user hinting that the object should be restructured. If the suggested change is accepted by the user the sub-module will move any methods to the bottom of the object declaration, during this operation various operations are made on the AST and with each subsequent modification a new AST is generated, this introduces a new problem of keeping track of other nodes that we may want to move. Fortunately the Roslyn API provides a way to keep track of the original node that was moved, from this we can track nodes that need to be moved, deleted or replaced without having to re-traverse the AST for similar matching nodes to those produced by the diagnostic from before.
